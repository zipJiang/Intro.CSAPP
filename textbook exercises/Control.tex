\documentclass[10pt a4paper]{article}
\usepackage{amsmath}
\usepackage{listings}
\usepackage{geometry}
\usepackage{engord}
\usepackage{multirow}
\usepackage{cases}
\geometry{left=2.0cm, right=2.0cm}
\lstset{language=C}
\lstset{breaklines}
\lstset{extendedchars=false}
\title{Homework 3}
\author{Tony Jiang\\1500012709}
\date{\today}

\begin{document}
\maketitle
\section{\textbf{3.60}}
  \begin{lstlisting}
    long loop(long x, long n) {
      long result = 0;
      long mask;
      for(mask = 1; mask != 0; mask = mask << n) {
        result |= x & mask;
      }
      return result;
    }
  \end{lstlisting}
  \subsection{\textbf{A.}}
    \textbf{mask} in \textit{\%rdx}; \\
    \textbf{x} in \textit{\%rdi}; \\
    \textbf{n} in \textit{\%rsi}; \\
    \textbf{result} in \textit{\%rax};

  \subsection{\textbf{B.}}
    The initial value of \textbf{result} is \textit{0}; \\
    THe initial value of \textbf{mask} is \textit{1};

  \subsection{\textbf{C.}}
    The test condition is whether \textbf{mask} equals zero.

  \subsection{\textbf{D.}}
    Every time \textbf{mask} is shifted left \textbf{n} bits.

  \subsection{\textbf{E.}}
    The answer already given above.

  \section{\textbf{3.61}}
    \begin{lstlisting}
      long cread_alt(long *xp) {
        long t = 0;
        long *p = xp ? xp: t;
        return *p;
      }
    \end{lstlisting}

  \section{\textbf{3.64}}
    \begin{equation*}
      \begin{cases}
        R &= 56 \\
        S &= 5 \\
        T &= 13
      \end{cases}
    \end{equation*}

  \section{\textbf{3.68}}
    \begin{equation*}
      \begin{cases}
        A &= 15 \\
        B &= 3
      \end{cases}
    \end{equation*}

  \section{\textbf{3.69}}
    \subsection{\textbf{the value of \textit{CNT}}}
    $$CNT = 7$$

    \subsection{\textbf{the declaration of \textit{a\_struct}}}
    \begin{lstlisting}
      struct a_struct {
        long n;
        long x[4];
      }
    \end{lstlisting}
\end{document}
